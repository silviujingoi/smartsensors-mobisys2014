\section{Experimental Setup}
\label{sec:experimentalSetup}

While our robotic testbed allows for live experiments, we chose
instead to use the robot to collect traces and to compare sensing
implementations using trace base simulation.  We opted for this
approach for several reasons.  First, it took the robot close to an
hour to complete a single experiment.  A thorough exploration of the
configuration space of the various sensing approaches would have
required over a year of continuous live experiments.  Moreover, taking
fine grain power consumption measurements while the robot is in motion
is not trivial.

We collected our traces by having the robot perform multiple runs with
a prototype smartphone attached to it back.  The smartphone ran an
application that kept the device always awake and continuously recorded
accelerometer readings for all three axises.  Each run generated a
trace that included timestamps for the start and end of each action
performed by the robot (which we use as the ground truth for our
experiments) and a list of timestamped acceleration readings.

In each run, the robot performed five different actions: standing
idle, walking, sit-to-stand transitions, stand-to-sit transitions, and
headbutt.  We created runs with three different levels of activity.
Runs in groups 1, 2 and 3 spent 90\% , 50\% and 10\% of the time
standing idle, respectively. The reminder of the time was allocated as
follows: 73\% for walking, 24\% for transitions between sitting and
standing, and 3\% for headbutts.  This set-up allows us to experiment
with detecting actions that are common, somewhat frequent, and rare.
In total, the robot executed 18 different runs: 9 for the group 1, 6
for group 2 and 3 for group 3.  We generated more runs for groups 1
and 2 because of the lower activity levels compared to group 3. To
eliminate bias, the list of actions was generated randomly for each
run, based on the expected probabilities of each action occurring.

The simulator evaluated each trace for each of the following sensing
configurations:

\textbf{Always Awake} The applications keep the phone awake all the
time constantly collecting accelerometer readings.  This setup
achieves the highest detection
recall~\footnote{$Recall=TruePositives/(TruePositives+FalseNegatives)$}
and
precision~\footnote{$Precision=TruePositives/(TruePositives+FalsePositives)$}
and provides a baseline for comparison.

\textbf{Duty cycling} We modified the applications so that they check
sensor readings periodically and then put the phone to sleep.  On
wake-up, the phone is kept awake for 4 seconds in order to collect
sensor data.  If an action is detected, the phone is kept awake for
another 4 second, and goes to sleep otherwise.  This software only
implementation runs on any mobile device and does not require special
hardware support.

\textbf{Batching} This version of the applications leverage the
low-power processor to collect and batch accelerometer readings while
the main processor sleeps.  The phone wakes up periodically, reads the
batch of sensor readings, runs the detection algorithm and goes back
to sleep.

\textbf{Activity} This configuration emulates hardware support for
recognizing a limited number of predefined activities, such as is
available on the MotoX.  For this purpose, we hardwired the
smartsensor as a {\em step detector} (EMA filter with alpha = 50\% and
x-axis > 2 $m/s^2$).  The  applications register an even handler for
the {\em step activity} and put the phone to sleep.  On wake-up, the
applications acquire a batch of sensor readings from the smartsensor,
run their detection algorithm, and go back to sleep.

\textbf{Filter} We let application define their own custom wake up
condition by configuring the filter and threshold used by the
smartsensor.  The applications register an even handler for the custom
event of interest and put the phone to sleep.  On wake-up, the
applications acquire a batch of sensor readings from the smartsensor,
run their detection algorithm, and go back to sleep.

\textbf{Oracle} A hypothetical ideal implementation that only wakes up
when an event of interest occurs.  Such a wake-up condition would
achieve the same detection precision and recall as Always Awake, with
the lowest possible energy consumption. The difference between the
power consumption of this method and the method using custom
filter-based wake-up conditions gives an upper bound on the potential
benefits of fully configurable hardware, beyond limited data filter
selection.

For each sensing strategies, and determines the the amount of time the
smartphone is awake and asleep and the total number of wake-ups.  In
addition, for each application the simulator determines its recall and
precision.  Finally, using an energy model derived from measurements
of our smartsensor prototype, the simulator estimated the average
power consumption of each configuration.  For {\em Always Awake} and
{\em Duty Cycling}, the power model accounts only for the energy
consumption of the Nexus 4.  For {\em Batching} and {\em Activity} the
model also includes the cost of the simple TI MSP430 micro-controller.
Finally, {\em Filter} and {\em Oracle} include the cost of either the
TI MSP430 or the TI Stellaris LM4F120H5QR, depending on the condition
being evaluated.








