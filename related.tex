\section{Related Work}\label{sec:related}

The idea of waking up a device when an event of interest occurs has been 
around since the inception of mobile phones. The phone's radio 
transceiver wakes up the device when an incoming call or a text message 
is received~\cite{gobi}. Wake on Wireless~\cite{shih2002wake} extended this idea by 
augmenting a PDA with a low-power radio that would send a wakeup message 
when an incoming call is received. Similarly, Wake on WLAN
~\cite{mishra2006wake} allows remote wakeup of wireless networking
equipment.

Turducken~\cite{turducken} generalizes the ``wake on event of interest'' 
approach to several types of applications and to multiple components 
operating at increasingly small power-levels. Little Rock~\cite{littlerock} 
applies Turducken's multi-tiered architecture to sensing on mobile 
devices. Reflex ~\cite{reflex} complements the idea proposed by Turducken 
by providing a shared memory abstraction to be used by the different 
processors.

Smartsensors differs from these approaches by hiding the heterogeneous 
nature of the system from the application developer. Creating an 
application that makes use of one or more Smartsensors does not require 
the developer to create any code that will run on the low-power processor(
s). We limit the available interface in order to increase portability, 
while still achieving the majority of the potential power savings.

Recently, smartphone manufacturers have started incorporated low-power 
processors into their architectures, but have only implemented limited 
APIs that provide fixed functionality. The Google Nexus 5 allows batching 
of sensor readings~\cite{ android4.4,nexus5}, and the Motorola Moto X 
provides recognition for a small number of predefined activities that can 
be used as wake up conditions~\cite{motox}. While these wake-up 
conditions work well for some applications, they are inefficient for many 
other types of applications that are not interested in the set of 
predefined activities.

Most of the previously noted works focused on system architecture 
modification in order to lower the cost of sensing. Alternative 
approaches have also been explored. Ace~\cite{ace} is a middleware that 
supports continuous context-aware applications while mitigating sensing 
cost for acquisition of context attributes (such as AtHome and IsDriving). 
It achieves power savings when multiple applications request strongly 
correlated context attributes. Additionally, it can reduce power 
consumption when a ``cheaper'' sensor exists, which can determine the 
value of a different context attribute that has a strong correlation with 
the requested context attribute (e.g. use the accelerometer to check if 
the user is jogging instead of using the GPS to determine if the user is 
at work). A middleware such as Ace is a great example of a library that 
can run on top of a Smartsensor architecture (once we implement support 
for multiple sensors) and achieve additional power savings. Sensor fusion 
has also been an active focus of related research. Data from multiple 
sensors can be used to increase context-awareness in mobile devices 
\cite{gellersen2002multi,biegel2004framework}.

While our focus was on power-efficient acquisition of sensor data, next 
generation mobile perception applications face related problems regarding 
partitioning of application code. MAUI~\cite{maui} enables fine-
grained energy-aware offload of mobile application code to remote 
servers. Similarly, Odessa~\cite{ra2011odessa} uses code-offloading to 
address the issue of processing excessive amounts of sensor data on 
resource constrained mobile devices.