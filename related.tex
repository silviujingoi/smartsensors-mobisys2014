\section{Related Work}\label{sec:related}

The idea of waking up a device when an event of interest occurs has been 
around since the inception of mobile phones. The phone's radio 
transceiver wakes up the device when an incoming call or a text message 
is received~\cite{gobi}. Wake on Wireless~\cite{shih2002wake} extended this idea by 
augmenting a PDA with a low-power radio that would send a wakeup message 
when an incoming call is received. Similarly, Wake on WLAN
~\cite{mishra2006wake} allows remote wakeup of wireless networking
equipment.

Turducken~\cite{turducken} generalizes the ``wake on event of interest'' 
approach to several types of applications and to multiple components 
operating at increasingly small power-levels. Little Rock~\cite{littlerock} 
applies Turducken's multi-tiered architecture to sensing on mobile 
devices. Reflex ~\cite{reflex} complements the idea proposed by Turducken 
by providing a shared memory abstraction to be used by the different 
processors.

Smart Sensors differs from these approaches by hiding the heterogeneous 
nature of the system from the application developer. Creating an 
application that makes use of one or more Smart Sensors does not require 
the developer to create any code that will run on the low-power processor(
s). We limit the available interface in order to increase portability, 
while still achieving the majority of the potential power savings.

Recently, smartphone manufacturers have started incorporated low-power 
processors into their architectures, but have only implemented limited 
APIs that provide fixed functionality. The Google Nexus 5 allows batching 
of sensor readings~\cite{ android4.4,nexus5}, and the Motorola Moto X 
provides recognition of a small number of predefined activities that can 
be used as wake up conditions~\cite{motox}.

Most of the previously noted works focused on system architecture 
modification in order to lower the cost of sensing. Alternative 
approaches have also been explored. Ace~\cite{ace} is a middleware that 
supports continuous context-aware applications while mitigating sensing 
cost for acquisition of context attributes (such as AtHome and IsDriving)
. It achieves power savings when multiple applications request strongly 
correlated context attributes. Additionally, it can reduce power 
consumption when a ``cheaper'' sensor exists, which can determine the 
value of a different context attribute that has a strong correlation with 
the requested context attribute (e.g. use the accelerometer to check if 
the user is jogging instead of using the GPS to determine if the user is 
at work). A middleware such as Ace is a great example of a library that 
can run on top of a Smart Sensors architecture (once we implement support 
for multiple sensors) and achieve additional power savings.

While our focus was on power-efficient acquisition of sensor data, next 
generation mobile perception applications face a related problems that 
are the active focus of other research. MAUI~\cite{maui} enables fine-
grained energy-aware offload of mobile application code to remote 
servers. Similarly, Odessa~\cite{ra2011odessa} uses code-offloading to 
address the issue of processing excessive amounts of sensor data on 
resource constrained mobile devices.

\iffalse
Over the past few years there has been a significant 
amount of work on reducing energy consumption in mobile 
devices. Reflex~\cite{reflex} achieves energy efficiency 
by leveraging architectural asymmetry between two low-
power processors and a powerful central processor. 
However, an application developer has to be aware of the 
heterogeneous nature of these platforms and even simple 
applications have to be refactored into distributed 
programs. The programmer has to divide the application 
into separate modules what run on the central and 
peripheral processors, respectively. Turducken~\cite{
turducken} is a mobile device architecture that 
integrates several mobile computing platforms that 
operate at different power levels into a multi-tiered 
device. The authors present two options available to 
developers in order to create distributed applications. 
One option is to ``use a proxy-based architecture that 
can take advantage of existing distributed application 
components''. Alternatively, the developer has to build 
an application that is customized for the system, by 
creating components for each tier. Smart Sensors differs 
from these approaches by hiding the heterogeneous nature 
of our system from the application developer. Creating an 
application that makes use of one or more Smart Sensors 
does not require the developer to create any code that 
will run on the low-power processor.

Little Rock~\cite{littlerock} proposes a system 
architecture similar to ours in order to run a basic 
Pedometer application. The paper presents two approaches 
to reducing energy consumption. In the first approach, 
the entire Pedometer application is executed on the low-
power processor, achieving significant power savings. 
This is possible because their implementation of the 
pedometer application only performs several algebraic 
and comparison operations for each accelerometer 
reading, hence it can run on the peripheral processor in 
real-time. The Steps application we implemented for our 
evaluation uses a more complex algorithm, including 
Discrete Fourier Transforms and Inverse Discrete Fourier 
Transforms. In the second approach, they propose a 
hybrid application, where the low-power processor is 
responsible for acquiring sensor readings, waking up the 
main processor, transferring the data and letting the 
phone perform all the processing. This approach is 
similar to the Batching method described in our paper 
and suffers from the same drawbacks. The power profile 
of their hybrid implementation is similar to our results 
for the Batching method. Similarly to Reflex and 
Turducken, Little Rock's heterogeneous architecture 
makes application development more complex. Smart 
Sensors is complementary to these works by introducing a 
programming model based on customizable data filters and 
wake-up conditions that simplifies application 
development while taking advantage of the power savings 
realized by the use of a low-power processor.
\fi
