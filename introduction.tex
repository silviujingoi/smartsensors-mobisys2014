
\section{\label{sec:Introduction}Introduction}

Today smart phones and tablets are used primarily to run interactive
foreground applications, such as games and web browsers.  As a result,
current mobile devices are optimized for a use case where applications
are used intermittently during the day, in sessions that last for
several minutes.  For example, Alice takes a little break to pick up
her phone, check the weather, read the latest news story and play a
game for a few minutes, and then puts it away to go back to work.  To
maximize battery life, current mobile devices are engineered to go
into sleep state for most of the day when they are not supporting
such interactive usage.

Unfortunately, most mobile platforms are a poor match for a growing
class of mobile applications that perform continuous background
sensing.  Examples range from context-aware applications~\cite{baldauf2007survey,hong2009context}, such as
targeted advertising, and medical applications~\cite{preuveneers2008mobile,tsai2007usability,hameed2003application} that improve our
well-being or even save lives using activity recognition (e.g., fall
detection), to applications that use participatory sensing to get a
better understanding of the physical world, such as noise pollution~\cite{maisonneuve2009citizen,maisonneuve2009noisetube}
monitoring or traffic prediction~\cite{hull2006cartel}.  While the processing demands of
these applications are modest most of the time, they require periodic
collection of sensor readings, which prevents the device from going to
sleep for extended periods of time.  As a result, applications that
perform continuous sensing may cause the device's battery to drain
within several hours.

To improve support for continuous sensing applications the research
community has proposed the use of fully programmable heterogeneous
architectures~\cite{reflex,turducken}.  In these approaches, developers
partition their applications to offload the initial filtering stages
of the application to a low-power processor (or a hierarchy of
processors).  When the code running on the low power processor detects
the occurrence of event of interest, it proceeds to wake up the phone
and passes control to the rest of the application.  While the ability
to run custom code on the low-power hardware provides great
flexibility, the significant complexity inherent in this approach has
so far prevented its adoption in commercial devices.  Instead,
smartphone manufacturers, realizing the potential of sensing
applications, have recently incorporated low-power processors into
their architectures, but have limited application developers to APIs
that provide fixed functionality, by either batching sensor readings
or recognizing a small number of predefined activities that can be
used as wake up conditions~\cite{android4.4,motox,nexus5}.

In this paper we argue that these APIs are insufficient for supporting
a rich and flexible set of continuous monitoring applications.
Batching is inefficient for applications that depend on infrequent
events and is not appropriate for applications that require crisp
response time.  An activity recognition API provides little
flexibility with no support for applications interested in events that
are not covered by the set of predefined activities.  Instead, we
argue that the API should expose access to filter-level functionality.
We present {\em Smart Sensors}, an approach that enables programmers
to create custom wake up conditions by choosing among a set of
predefined filtering algorithms and tuning their parameters.  By
providing developer with access to lower-level filters that implement
commonly used algorithms (e.g., FFT, exponential average), as opposed
to higher-level activity detectors, smart sensors provide a better
balance between flexibility and ease of deployment.  Smart Sensors allows
developers to create a broad set of wake up conditions that can be used
to detect a wide range of activities.  Simultaneously, they are
significantly easier to program compared to fully programmable
offloading.  Moreover, since the filters are pre-specified, their
implementations can be optimized for each low-power processor,
improving application portability between devices.  Furthermore, it is
possible to combine the filtering and wake up operations when
applications register interest in the same set of events.  While smart
sensors could be programmed directly by individual applications, it is
easy to imagine user-level activity recognition libraries that
encapsulate the functionality of the smart sensor to provide simple
wake up conditions for a large number of activities.

To evaluate the benefits of our approach, we have developed a prototype
implementation that extends a Nexus 4 phone with a low-power sensor
board.  Our experiments with various accelerometer-based applications
show that: 
\todo[inline]{TODO: summarize results.}

The rest of this paper is organized as follows.
Section~\ref{sec:background} discusses existing approaches to
continuous sensing and their limitations.  Section~\ref{sec:approach}
introduces our new approach for continuous monitoring, which lets
developers configure the filters used to implement wake up conditions.
Section~\ref{sec:prototype} describes our Nexus 4-based prototype.
Sections~\ref{sec:experimentalSetup} and~\ref{sec:results} present our
experimental setup and the results from our evaluation.  Finally,
Sections~\ref{sec:related} and~\ref{sec:conclusion} describe our work
in the context of related work and conclude the paper.

