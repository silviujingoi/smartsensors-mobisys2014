\section{Conclusion}
\label{sec:conclusion}

In this paper we proposed Smartsensors, an approach that uses a low-power processor to perform sensor data acquisition and wakes up the main processor when an event of interest occurs. The API provided to developers hides the heterogeneous nature of the system, ensuring application development does not increase in complexity. The use of an API also allows Smartsensor-based mobile applications to be portable to other Smartsensor-enabled devices. Our experiments showed that the use of SmartSensors can match the detection recall of an approach that keep the phone constantly awake and reduce average power consumption by up to 94\%. In most usage scenarios we explored, we were able to achieve over 95\% of the power savings that would be achieved by a ``perfect'' wake-up condition. This suggests that a system that allows custom code offloading to the low-power processor can achieve minimal additional power savings.

Our immediate future work includes developing Smartsensors based on other sensors available on mobile devices, such as the microphone, camera or location sensor. We believe application developers may face challenges in selecting the optimal data filter and wake-up parameters for their application. We plan on creating self-learning algorithms that, given feedback from the application about events of interest, will be able to determine the optimal data filter and wake-up parameters. Additionally, we will be exploring possible optimizations for scenarios where multiple applications are interested in similar sensor data. We are also interested in using sensor fusion to create even ``smarter'' sensors that enable the development of context-awake mobile applications.
