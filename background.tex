
\section{Background}
\label{sec:background}

Mobile application developers can choose among several existing approaches when
attempting to perform continuous sensing. The availability of some of these
approaches depends on the mobile device's hardware and operating system.

\subsection{Always Awake}

The basic approach for continuous sensing is to keep the mobile device
\emph{always on}. As an example, on the Android operating system, the sensing
application acquires a wake lock to prevent the device from sleeping and
registers a listener for a specific sensor. Whenever the sensor produces a new
reading, the operating system passes the reading to the application via a
callback function. This operation continues until the listener is unregistered
or until the wake lock is released. The benefit of this approach is that it
achieves the highest possible accuracy because all the sensor data is delivered to the
application. However, the device cannot a sleeping state and hence the power
consumption of this approach is high.

\subsection{Duty Cycling}

To avoid keeping the device awake for extended periods of time, \emph{duty 
cycling} allows a device to sleep for fixed, usually regular, periods of
time. An application on Android can implement duty cycling by scheduling a wake-up alarm using the system's alarm service and by ensuring that it has not
acquired a wake lock. When no application has a wake lock, the operating system
can power down the processor. When the wake-up alarm fires, the processor wakes up and notifies the application. Next, the application collects sensor readings
for a short period of time, and then reschedules another alarm.

Duty cycling has several drawbacks. While the device is sleeping, events of interest will not be 
detected because the application does not receive any sensor data that were produced at those 
times. Additionally, during the short time periods when the device is awake and collecting 
sensor readings, the events of interest might not occur. These power transitions are wasteful 
and expensive (as we show later). 

The basic duty
cycling approach can be improved by dynamically adjusting the wake-up time. When
an application is woken up and detects an event of interest, it continues to
collect sensor readings until events do not occur for some period, after which
it returns to its regular duty cycling schedule. This improves detection
accuracy (e.g., event recall) but at the expense of increased energy
consumption. Unfortunately, finding a good balance between accuracy and power
savings depends on the specific application and the user's behaviour. Another
drawback of this approach is that is does not scale well with multiple
applications. It is possible for mutually unaware applications to implement
conflicting wake-up schedules, resulting in little or no sleep.

\subsection{Sensor Data Batching}

An extension to duty cycling is \emph{sensor data batching} in which
the device hardware is able to collect sensor readings while the main
processor is asleep. Upon wake-up, the entire batch of sensor data is
delivered to the application(s) that registered a listener for that
specific sensor. Unlike duty cycling, batching requires hardware
support (e.g., it is currently only available on the newly released
Google Nexus 5, running Android 4.4), but it enables applications to
receive all the sensor data, and hence provides detection accuracy
similar to the Always Awake approach. However, batching affects
detection timeliness. Applications may detect events of interest many
seconds after they actually occurred. Also, as with the duty cycling
approach, the device may wake up to find out that no events of
interest have occurred in the current batch, which is wasteful.

\subsection{Computation Offloading}
\label{subsec:computationOffloading}

Generalizing batching, a low-power peripheral processor can be used to
collect and \emph{compute} on sensor data while the main processor is
asleep~\cite{reflex,turducken}.  This approach allows developers to
offload their own sensor data analysis algorithms to the peripheral
processor, enabling arbitrarily rich sensing applications. However,
developers are exposed to the full complexity of the architecture,
making application development more difficult. Even simple
sensor-driven applications have to be refactored into distributed
programs, and the code that is offloaded has to be chosen
carefully so that it can run in real time. The latter is complicated
because it depends on the type and functionality of the peripheral
processor that is available. As a result, multiple versions of the
application may be required to handle hardware variations across
phones. Furthermore, since each application is written independently,
this approach makes it hard for multiple applications to program or
use the same sensor.

%% The high level of programming complexity is very likely a cause why such an
%% architecture is not supported by current commercial systems.
%% We have the same problem -Ashvin.
% it appears Reflex provides an API to get sensor data, so devs don't need to
% write code for each sensor version

\subsection{Activity Detection}

Given the challenges with full computation offloading, recent mobile
devices limit offloading to a small, predefined set of \emph{activity
  detection} algorithms. For example, the Motorola Moto X~\cite{motox}
uses two low-power peripheral processors to wake up the device when
certain events occur. A dedicated natural language processor is used
to wake up the device when a user says the phrase ``OK Google Now''.
Also, a contextual processor is used to turn on a part of the screen
when the mobile device is taken out of a pocket, or start the camera
application when the processor detects two wrist twists. Additionally,
the latest Android 4.4~\cite{android4.4} supports specific wake-up
events based on detecting ``significant motion'' or a step, if a device
has appropriate sensors. However, the application developer has no
control over the underlying activity detection algorithms, including
customization of parameters. This approach is easy to use, and it
provides good energy savings because the device only wakes up when the
event of interest occurs. However, it only works for applications that
can take advantage of the predefined set of activities, limiting the
number and types of applications that can be supported.

%% The main advantage of this approach is it ease of use application developers,
%% provided that the platform supports the activities of interest. Also, very
%% good energy savings are achieved as

%% Fundamentally, this approach provides callbacks for pre-defined activities
%% such as walking or specific phrase recognition.

%% This approach works well if the application developer is interested in one of
%% the predefined activities but . However, the approach provides no support for
%% applications that are interested in activities that are not part of the
%% pre-defined set.  this is redundant. -Ashvin

