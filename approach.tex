
\section{Approach}
\label{sec:approach}

The greatest challenge we faced was determining what computations should be executed on the low-
power processor. On one hand, pushing more computation to the low-power processor will result in the
main processor remaining asleep for longer periods of time and increasing energy savings. On the other
hand, the throughput of a low-power processor is much lower than that of the main processor. We want
that any sensor data analysis performed by the low-power processor to be executed in real-time and the
main processor to be woken up as soon as necessary. Additionally, there are many parts of a mobile
application that cannot be executed on the low-power processor. Any code that handles the graphical
user interface or that needs to use resources not available to the low-powered processor needs to run
on the main processor.

A second challenge we are facing is determining what appropriate wake-up conditions look like. For this, we examined sensor-driven applications to see what type of computations are performed on the sensor data. By analyzing such applications, we noticed that the computations performed on the sensor data follow a number of steps that are application-specific. However, at each step, the application may be looking for a specific event. For example, an
accelerometer-based fall detector is looking for acceleration representing free-falling, followed by a big
spike in the acceleration magnitude TODO: add footnote defining magnitude   representing the moment of
impact TODO: add citation. We also noticed that some of the initial steps may be common among several
applications. For example, several accelerometer-based applications initially check whether the
acceleration magnitude exceeds a certain threshold in order to determine if there is any movement.
Also, because there exists a significant amount of noise in the acceleration data, many applications are
implementing low-pass filters to obtain a smoother form of the accelerometer readings. If multiple
applications are running at the same time, they are each checking for these events independently and it
might result in redundant computation. Common steps such as thresholding and low-pass filtering are
excellent candidates for functionality that should be considered as part of defining wake-up conditions.

We chose to focus our efforts on accelerometer-based applications because there exists a wide-range of
such applications available on the various mobile application markets. Applications using the device's
camera or microphone are fairly common as well. However, it is quite rare that multiple applications are
using these sensors simultaneously.
