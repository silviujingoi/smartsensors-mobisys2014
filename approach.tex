
\begin{figure}[p]
	\includegraphics[width=9cm]{android_architecture_current.png}
	\caption{Simplified architecture of Android 4.4 sensor system}
    \label{fig:androidArchCurrent}
\end{figure}

\begin{figure}[p]
	\includegraphics[width=9cm]{android_architecture_proposed.png}
	\caption{Proposed architecture based on Android 4.4 sensor system}
    \label{fig:androidArchProposed}
\end{figure}


\section{Approach}
\label{sec:approach}

Our aim is to enable low-power, continuous sensing tasks on mobile processors.
As described earlier, the current approaches for low power sensing trade ease of
programming with the generality of the approach. On the one hand, fully
programmable offloading of code to low-power processors allows developing
arbitrary applications [TODO:cite]. However, it raises several challenges, such as
requiring a split or complete redevelopment of application functionality, and
carefully choosing the code that can be offloaded so that it can run in real
time. The latter is complicated because it depends on the type and functionality
of the low-power processor that is available. Furthermore, since each
application is written independently, this approach makes it hard to optimize
power consumption for multiple sensing applications. Alternatively, the phone
can be designed to support a predefined set of activities but this approach
limits the number and types of applications that can be supported.

Our approach for low-power sensing takes a middle ground and provides a
filter-based sensor API for application programs. Applications choose from a
pre-specified set of event filters that are run by the low-power processor to
detect application-specific events of interest. When these events occur, the
main processor is woken up and the application code is invoked. The result is
that applications view the sensors as ``smart'' sensors that generate relevant
events only. This approach enables supporting a large number of sensing
applications while being significantly easier to program compared to fully
programmable offloading, as we show later. Since the filters are pre-specified,
their implementations can be optimized for each low-power
processor. Furthermore, it is possible to combine the filtering and wake up
operations when applications register interest in the same set of events.

The main challenge with our smart sensors approach is defining the appropriate
set of event filters for each sensor. First, the filters need to be mainly
computational tasks. Any code that needs to use resources not available to the
low-powered processor, e.g., the graphical user interface, needs to run on the
main processor. Second, there is a trade-off in the computational tasks that are
carried out by the low-power processor. Pushing additional computation to the
low-power processor will generally result in higher accuracy filters, and the
main processor will remain asleep for longer periods of time, thus increasing
energy savings. However, this computation needs to be performed on the low-power
processor in real time, and the main processor woken as soon as the event is
detected. We evaluate this trade-off between filter accuracy and power savings
by comparing the behavior of two different low-power processors.

We examine the types of computation performed by various sensing applications to
determine a common set of filters that are needed for these applications. In our
experience, while the application logic varies widely, their sensor event
detection algorithms are relatively simple and have commonalities. For example,
an accelerometer-based fall detector looks for acceleration that represents
free-fall, followed by a spike in acceleration~[TODO:cite]. Similarly, a step
detection algorithm looks for a local maxima within a certain range of
acceleration~[TODO:cite]. 

These applications often perform some common initial steps. For example, several
accelerometer-based applications initially check whether the acceleration
magnitude exceeds a certain threshold in order to determine if there is any
movement. Also, the acceleration data has significant noise, and hence many
applications implement low-pass filters to smoothen the accelerometer readings.
If multiple accelerometer-based applications are running at the same time, they
are each checking for these events independently, possibly resulting in
redundant computation. Common steps such as thresholding and low-pass filtering
are excellent candidates for event filters.

